Dark though the sky may be as storms pass and whiten the ground with falls of snow, there can be no doubt as Primilis moves forward that growth is beginning to gain a vast and increasing momentum urged on by the mounting sun.\blfootnote{The prose introductions to each month, detailing the flora and fauna one would have observed in England during the age in which the poetry was written, are taken from Miles Hadfield's excellent and learned \emph{English Almanac}, with a handful of amendments and excisions.}

It is in the woods that we see most signs of this. The canopy of branches traps radiation of the accumulated day's heat from the earth during cold, clear nights, so that the woodland soil becomes warm earlier than that in the open. Most of the woodland herbs, too, have to get their flowering finished before the leafing of the trees cuts the sunshine off from the forest floor. This is so with both Wordsworth's lesser celandine and wood anemone, which both flower freely and carpet open woods.

Our wood anemone is found growing over a wide area of the northern hemisphere. It is surely one of our loveliest flowers, formed from a ring of large sepals which are coloured by many shades, from white to pinks and blues, instead of the usual green. Another very common plant which breaks away from both the usual yellows and blues of early spring and the habit of early leafing is dog's mercury -- though the greenish flower-spikes look more leaf-like than floral. The god Mercury  discovered its medicinal properties which mortals have now long forgotten. This abundant but uninteresting-looking herb is related to our own spurges, and the weird euphorbias of other lands. Another striking exception to our spring rules is toothwort, which may now be found in flower under hazel trees, upon whose roots it lives. This obviates the need to grow its own green leaves. The flowers, in a heavy architecturally-designed spike, are a strange colour, usually described, not very accurately, as ``flesh''. They soon turn a sombre, purplish brown, and are handsome decorations. Another architectural beauty flowers too, the striking hellebore or setterwort, another all-green plant.

Many trees, mostly catkin-bearers, flower now. Unlike most herbs they do so before their leaves appear, on bare twigs. In the woods wych-elm, and in the hedgerows field-elm, will probably still be flowering. Their large-winged green fruit develop in a matter of days as soon as fertilisation has taken place. In some years this makes the elms look prematurely-leaved. Jolliest of all are several kinds of shrubby osiers, willows, or sallows. The male bushes bear golden, pollen-bearing catkins, and the female more modest silvery ones: golden palm and (the unfortunately-named) pussy-willow. Both are covered with bees on mild days.

The alder, too, is profusely catkined: long, pollen-bearing ones and little cone-like female flowers that become woody. In winter they open and the seed showers down and often floats to its destination. A fine old catkined alder, lit by the sun, overhanging a dark pool, is a lovely sight in its subtle colourings of grey and pinkish browns.

Other trees bearing catkins now are our little native poplar, the aspen, larches -- celebrated by Tennyson, abhorred by Wordsworth -- and, where it grows in the eastern counties, hornbeam, which looks rather odd, rather like a betasselled beech. Other poplars are also flowering. Undoubtedly the most handsome is our native black -- called birch-leaved -- poplar in its male form. This tree was once much planted along streams in flat park-land. Its heavy, black, open branches become covered with long crimson tassels. It is a brilliant and rather uncommon sight. Old trees grow top-heavy, and the branches fall apart and break up. They are removed eventually by energetic clearers of watercourses, and nobody thinks to replant them; they are quite useless trees, only beautiful.

In the hedgerows blackthorn will be showing very white against the dark plough, and in orchards, damson: early to flower and late to fruit.

So much for the trees. Of herbs in open places the coltsfoot will still be prominent, while dandelions and daisies are opening their long season. Stream and marsh flowers are mostly late bloomers, but surprisingly the marsh marigold starts now.

You may find other flowers, especially if the season has been mild. Those irrepressibles, chickweed and groundsel, are sure to be busy; ground-ivy -- pretty general -- a few dog violets, and wild pansies; the smaller speedwells and white deadnettle are sure to be found.

Suburbia becomes jolly. Crocuses, purple, white and orange, cover the ground. They are joined by the clear pink blossoms on the angular, open-branched almond tree. This was brought here, probably from Algeria, early in the sixteenth century; it has not become naturalised and would soon disappear if it were not continually propagated and planted by gardeners.

So much for some of the more obvious excitements that greet the botanist. Bird-watchers are kept equally busy. Robins, blackbirds, hedge-sparrows and thrushes are busy nesting. Blackbirds and chaffinches are two of the many birds now coming into full song. The great drumming contest gets well underway. It is, of course, between the supporters of different theories explaining the peculiar mechanical-sounding noises made by the greater spotted woodpecker as he clings to a tree and the snipe as he swoops through the air.

Migration becomes an important object of study as the spring movements of birds increase. Most welcome are our own early arrivals, such as the wheatear. Some places will seee chiff-chaffs at the end of the month, and rather fewer flocks of yellow wagtails. Possible arrivals are willow-wrens and sand-martins. Aristocratic geese and many of the ducks leave us; redwing and brambling move to the north. Puffins cease their wanderings and collect in crowds at their breeding places. At this season of the year they grow their fancy noses, which disappear in the autumn. Most seaside visitors imagine this decoration to be permanent.

The lighter sleepers awake. Fine days bring out queen wasps. The boom of a queen bumble-bee as she starts her energetic year will take the mind forward to hot summer afternoons when the only sound seems to be a murmur of bees' wings. Honey-bees are, of course, out on every fine day searching crocus and sallows.

Both kinds of squirrel will be finishing their nests -- or ``dreys'' -- builing in the tops of trees. Badgers are spring-cleaning, turning out the bedding that has kept them warm during the winter and preparing new chambers in their deep sets, which they line with fern and grass ready to receive a new generation of ``earth pigs''.

Most kinds of bat leave their belfries and crevices and take a flight soo after the middle of the month. Lizards awake from their sound sleep and pair. Grass snakes may be out on a fine day. Toads will start their wanderings, and the more active frog may already have started breeding and spawning.

This is the time for hares to go mad. The buck cavorts and plays wild antics as he pursues the doe. The urge of spring makes the peaceful mole ferocious and quarrelsome. Both he and the little shrew will now fight their own kind to the death.

This is also the time for cubs to be born to foxes and whelps to otters. The stags of red deer drop their antlers.

In the fish world miller's-thumb, perch and pike are spawning, the earliest fish to do so.

A small tortoise-shell butterfly, distracting the attention of the congregation as it flaps in the sunshine on a church window, is often the first reminder of the year that the supremacy of drab moths has ended. The husbands of those apertous, crawling females are soon forgotten when we see the first tortoise-shells, brimstones and peacocks, three butterflies that hibernate so that they can begin the breeding season early. A few cabbage whites usually appear, as well as some slightly jollier moths, such as the yellow underwing. Manuals inform collectors that towards the end of the month they may start sugaring and visiting catkined sallows with a lantern, but warn them that their yields may be meagre. On the other hand, hibernating larvae are getting active, and butterfly breeders may find a good number of caterpillars.
