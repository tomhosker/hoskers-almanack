This month we can be in no doubt, however warm and sunny the weather, that summer is ending. Everywhere we notice fruits -- tufts of down on the composites, berries in the hedgerow, pears and apples in the orchard -- all proper to autumn and winter. Even in this month greengages will have ripened, plums will be at their best, some apples will be ready, and William's Bon Chr\'etien pear will be picked.

The woodsides and hedgerows are a tangle. We have rose hips, thorn haws, and hazel nuts; below them are the red berries of cuckpp-pint. The whole mass is overgrown by twiners and stagglers, all in fruit. In the dense mass two are almost universal -- blackberry and black bryony.

Blackberries -- they must be eaten before Michaelmas Day (the twenty-ninth day of September in the Gregorian calendar) when the devil gets into them -- have long been studied by botanists who loved to argue over the varying forms they found. Genetecists have now found that attempts to reach finality by the older methods of botanical classification may only lead to bewilderment. The extraordinary variation within the group of plants known as \emph{Rubus fruticosus} is due to hybridisation and involved complexities of fertilisation and seed production. It will be remembered that the long shoots of blackberries generally root when they reach the ground, so that a variation, having once come to maturity, will probably increase and perpetuate itself quite independently of any seed that it produces. Thus we have a ``clone'' both originated and propagated by nature; usually nature (man often claims to have helped) does the originating and man (sometimes even impeded by nature) does the propagating.

Black bryony is a much simpler plant to study. It is the sole English member of the yam family, which is largely tropical. The only confusion is in its name. Black applies to the root, not the berry, which is scarlet. It is no relation of white bryony. This has a red or orange berry, and is named from the thick white root which is called by some English mandrake, and by them is claimed to have the powers of that magic plant. White bryony is plentiful in many places, though it has not the same wide distribution as the last two plants.

On limestone soils hedges will often be mounded over with a mat of the silky tassels of our only clematis. In flower it is called traveller's joy; in seed, old man's beard.

The dearth of wild flowers drives butterflies to our gardens. Buddleias are covered with them. The commonest are the usual whites, red admirals (who love rotting plums even more), tortoiseshells, peacocks, painted ladies and brimstones. At night many handsome moths come to the flowers. In some years the south coast has invasions of the rare striped hawk-moth, which feeds in the evenings on the nectar of petunias. But however crowded the garden may seem, butterflies and moths decrease in numbers very considerably during the month. Many that we see are about to hibernate.

Ivy flowers provide one of the few sources of food for nectar-sipping insects left among wild plants. Wasps and drone flies in particular swarm around them. Flies, mosquitoes, wasps, hornets, midges and all the more pestilential, stinging and irritating of Pandora's horde seem at their most vicious on a warm September day. A cheerful and harmless little fellow who also reaches his highest population this month is the water whirligig beetle.

September is also a month when we see or feel much of some insect-like animals. Droplets from heavy dews cling on to and display the webs that spiders spin over hedges and grass. Many kinds now achieve their maximum population and are busy egg-laying. The frosts will kill off most of them -- the common, and now very abundant, garden spider will be one of the first to die when cold weather comes. The large long-legged house spider, \emph{Tegenaria atrica}, in particular seems to be moving noisily about and mildly alarming us during early autumn. The irritating ``harvester'', which burrows into our shin for no good reason, and certainly not to its own benefit, is at its most irritating now, particularly where there are limestone or chalky soils. It would surely be better for everybody if it remained among the stubble and bracken. Those strange, long-legged creatures, superficially spider-like, of the order Opiliones, known as harvestmen, are also often seen. Little is known of them or their way of life.

Woodland and inland birds leave in large numbers. Some nightingales may have left in Sextilis, and the rest now follow. Early in the month, garden, grasshopper and reed warblers, as well as redstarts, leave. Then whitethroats and whinchats. Towards the end we lose willow warblers, chiff-chaffs, common sandpipers, spotted flycatchers, tree pipits, wheatears, turtle-dove and yellow wagtails. Swallows and martins congregate, lining telegraph wires and perching on dead trees, ready for their flight.

Arrivals are teal, pintail, jack-snipe and countless waders. The big grey lag geese, which once bred here much more freely than they now do, fly southward to us. If the weather is bad in northern Europe the first redwing and fieldfare may arrive.

But for the robin and twittering of the accumulating swallows, there is little bird song. Enthusiastic sparrows and wood-pigeons may attempt a final nest.

In the longer nights we may now hear the barking of foxes. The strange (it has well been called peacock-like) yelping of the vixen and the sharp yapping of the dog make an unearthly din.

Other migrations are under way. Some eels, specially dressed in a silver travelling livery, no doubt to match the `golden lamps in a green night', start on their way to Bermuda. There, in one little spot of the whole vast ocean, the long fantastic journey completed, they breed.

In the reverse direction salmon and those trout visiting the sea will be entering our river-mouths and making their astonishing climb up to the fresh waters where they spawn.

Of wild flowers heather and the yellow composites are among the few that persist. The lovely blue chicory, a rather late flowerer, often surprises us in those districts where it grows on the roadsides and edges of fields. The Compositae also give us most of the September garden flowers -- Michaelmas daisies, dahlias and chrysanthemums. These flowers -- purples, golds, crimsons, yellows -- are the signals warning us that October, and real autumn, is very near.
