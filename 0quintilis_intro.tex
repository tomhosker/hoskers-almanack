As if to balance the richness and variety of vegetation, birds become quiet in Quintilis and sing no more. The woods become silent as, perhaps, life becomes easier for them and, ragged and dull, they go into partial retirement. Old cuckoos leave our shores for the south -- though the cuckoos departure is much less publicised than his arrival.

The cuckoo's departure is but one sign that bird movement is under way again; the waders are among the first to be on the move. Most birds having finished nesting are dispersing around the countryside, and many kinds may be watchd in quick succession as they pass along hedgerow or through woodland. Quintilis is the month when crossbill invasions occur.

The larger and stronger plants, those able to fight their way through the lush growth of grasses, now flower in the hedgerows. Meadow-sweet and valerian are to be found in the damp places, and round them hovers the demoiselle -- the `blue thread loosened from the sky.' Elsewhere along the roadside coarser umbellifers abound, handsome structures mounted with their umbrellas of dingy-white flowers which, as often as not, are covered with colonies of little beetles and strange flies. Honeysuckle is in the woods, and our more magical herbs are represented by the St John's worts and enchanter's nightshade, for which Linnaeus, displaying his usual perfect taste in apt selection, chose the lovely name of \emph{Circaea lutetiana}, which, from botanical Latin, may, I suppose, be translated into ``The Enchantress from Paris''.

The blue forget-me-nots are everywhere. The old name of scorpion-grass may hint at the similarity of the uncurling flower stem to a scorpion's tail, but how innocent of sting or anything other than purity it is when uncurled and opened ready, we imagine, to give the fatal stab!

This is one of the smaller flowers now open. Others are to be seen on the heaths and hills. The pink glow of the heathers, the delicate harebell, and, where there is limestone, the yellow rock-rose opens in the sunshine for an hour or two before petals tumble away.

Around the edges of pools water plantain, arrow-head, and rushes form, with their clear-cut geometric designs, a margin to the dark mirrors on which float water-lilies -- the white one so beautiful that it is dedicated to the nymphs, and the yellow ending its days as a comic brandy bottle. In damp spots there are pink patches of bog pimpernel, and, in some places, the exciting sundew is hard at work catching flies while its flower stems uncurl.

Indeed, the end of Quintilis is the time when the marsh flowers come to their best; the unblossom-like flowers of the bulrush \emph{Typha latifolia} are seen, and also those of yellow monkey-flower or mimulus -- an American that has become naturalised in many damp places, and is related to the musk that so mysteriously lost its scent.

It is the month of the great exodus of the little frogs, no longer admitting that they were once tadpoles, to their ponds. They are often so thick upon the ground that it is impossible to avoid walking on them, and millions must perish when their march to dry land happens at the same time as the mowing of grass.

Two famous weeds, obnoxious, but not without beauty, become ubiquitously prominent towards the end of the month. Both seem to be hand-in-glove with man and his doings, particularly in the waste places that he makes. The first of these to gain a reputation was the ragwort, \emph{Senecio Jacobaea}. With brassy daisy-flowers and fancifully cut, even tattered leaves, it can be seen in every dry desolation, along roadsides, and particularly in poverty-stricken meadows. In any quantity it is poisonous to stock, as is its relative, groundsel, if given in too large doses to canaries. Both belong to the same family as the candlelike succulents of African mountains. Ragwort is a menace to farmers and an increasing nuisance everywhere.

The other pest is the graceful rose-bay willowherb, the townsman's fireweed. This lovely plant, its tall spikes of slender leaves ending in a spire of rose-pink flowers, thrives equally well in ground that man has cleared or left derelict, either in town or country. In autumn the long thin pod splits, and the two parts curl back and scatter innumerable seeds, each with a tuft of down, which float gently in clouds and spread all around. The young plant roots but lightly in the surface, and as soon as it is established sends out runners, each of which will send up another spike. It quickly spreads, forming a mat of choking vegetation. Its increase during recent times has been surprising, and is now alarming. Probably the clear felling of woodlands during and after the Fourteen-Eighteen War gave it a good start. It now colonises clearings so quickly and effectively that it is swamping and perhaps destroying the smaller plants that would usually grow in such a place -- probably altering the whole nature of the vegetation. It has spread to spoil-banks and ground that has been cleared of buildings, and seems especially happy in the wake of a fire, for the network of its roots thrive in the arid, crumbly surface that is left. In flower it is a lovely sight, and loved by the honey-bee for the the nectar it carries. The bee-keeper knows when his hives are working the wastelands, for his bees return dusty white from the fireweed's pollen.

The citizen has two other flowers this month. Wherever privet has escaped the clippers it bursts into blossom with rather drab little spikes that look like, and are, very poor relatives of the white lilac. The scent is heavy, sickly, and symptomatic of high summer in suburbia. It is liked by the bees. It is liked by the bees, but their keeper objects to the honey they make from it. The heavy flow of nectar produced by the lime on warm still days -- sometimes so freely that it can be shaken from the flowers -- is, however, a joy equally to the bee and his owner. In a good year many a town hive's crop is of its dark rich honey. The trees are often golden with the blossom, and the sweet scent of lime avenues, or even single trees, drifts along the city streets as a reminder of old lime walks in country churchyards and parks.

Another forest tree, the sweet or Spanish chestnut, is also in flower. It is often densely covered with white, spreading catkins, thick with pollen. At their bases hide the female flowers with their feeler-like stamens which catch the pollen, are fetilised, and soon develop into chestnuts.

The insect world has a great month. The song of the grasshopper is heard again on sunny days, and at night the first glow-worms may be seen. If the weather is right ants may swarm out of the nest, and the winged generation take to the air and mate. Butterflies abound.

The first home-bred generations of what we might call our own domestic or garden butterflies hatch out -- peacock, tortoiseshell, red admiral and painted lady -- sometimes while their immigrant parents are still on the wing; the humming-bird hawk-moth hovers in our gardens, and sometimes the newspapers get excited over invasions by the clouded yellow.

The gardener begins to harvest, as do the fruit-eating birds. (Perhaps that is why they are silent.)  There are now the finest cherries, red, white and black currants, raspberries, the last of the strawberries, and dessert gooseberries.

Vegetables abound, and though the seasons of the aristocratss of the flower garden is passing, the herbaceous borders are gay with the achievements of nurserymen; the sweet-scented white jasmine of the poets is covered with flowers.
