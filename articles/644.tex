\documentclass[MAIN]{subfiles}
\begin{document}

\settowidth{\versewidth}{\vin There's men from the barn \& the forge \& the mill \& the fold;}
\begin{verse}[\versewidth]
\poemlines{5}
The lads in their hundreds to {\sc Ludlow} come in for the fair;\footnotetext{Prof Alfred Housman (1859 -- 1936). \cite{oxfordlarkin}.}\\*
\vin There's men from the barn \& the forge \& the mill \& the fold;\\
The lads for the girls \& the lads for the liquor are there,\\*
\vin And there with the rest are the lads that will never be old.\\!

There's chaps from the town \& the field \& the till \& the cart,\\*
\vin And many to count are the stalwart, and many the brave,\\
And many the handsome of face \& the handsome of heart,\\*
\vin And few that will carry their looks or their truth to the grave.\\!

I wish one could know them; I wish there were tokens to tell\\*
\vin The fortunate fellows that now you can never discern;\\
And then one could talk with them friendly and wish them farewell\\*
\vin And watch them depart on the way that they will not return.\\!

But now you may stare as you like and there's nothing to scan;\\*
\vin And brushing your elbow unguessed-at and not to be told\\
They carry back bright to the coiner the mintage of man,\\*
\vin The lads that will die in their glory \& never be old.
\end{verse}

\end{document}
