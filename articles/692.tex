\documentclass[MAIN]{subfiles}
\begin{document}

\settowidth{\versewidth}{\vin They go to the fire; the nostril pricks with smoke}
\begin{verse}[\versewidth]
\poemlines{5}
Now is the time for the burning of the leaves.\footnotetext{Prof Laurence Binyon (1869 -- 1943). \cite{oxfordlarkin}. Prof Larkin calls these four verses `The Burning of the Leaves', although in other books they are simply the first of five parts of a longer poem of the same name.}\\*
\vin They go to the fire; the nostril pricks with smoke\\
\vin \vin Wandering slowly into a weeping mist.\\
Brittle \& blotched, ragged \& rotten sheaves.\\
\vin A flame seizes the smouldering ruin and bites\\*
\vin \vin On stubborn stalks that crackle as they resist.\\!

The last hollyhock's fallen tower is dust;\\*
\vin All the spices of june are a bitter reek,\\
\vin \vin All the extravagant riches spent \& mean.\\
All burns. The reddest rose is a ghost;\\
\vin Sparks whirl up, to expire in the mist: the wild\\*
\vin \vin Fingers of fire are making corruption clean.\\!

Now is the time for stripping the spirit bare,\\*
\vin Time for the burning of days ended \& done,\\
\vin \vin Idle solace of things that have gone before:\\
Rootless hope \& fruitless desire are there;\\
\vin Let them go to the fire, with never a look behind.\\*
\vin \vin The world that was ours is a world that is ours no more.\\!

They will come again, the leaf \& the flower, to arise\\*
\vin From squalor of rottenness into the old splendour,\\
\vin \vin And magical scents to a wondering memory bring;\\
The same glory, to shine upon different eyes.\\
\vin Earth cares for her own ruins, naught for ours.\\*
\vin \vin Nothing is certain, only the certain spring.
\end{verse}

\end{document}