\documentclass[MAIN]{subfiles}
\begin{document}

\settowidth{\versewidth}{\vin And I laid me down with a will.}
\begin{verse}[\versewidth]
\poemlines{5}
Under the wide \& starry sky,\footnotetext{`Requiem', Robert Stevenson (1850 -- 1894). \cite{ptmgmc}. These verses are inscribed, according to Stevenson's wishes, on his tomb on Upolu, an island now part of the Independent State of Samoa. Philip Larkin's infamous `This Be the Verse' is presumably a response to this poem. \P What has come to be regarded as the standard version of this poem gives the penultimate line as, `Home is the sailor, home from sea', i.e. without the second \emph{the}; however, the Almanackist prefers the version with said \emph{the}, and, indeed, this is the version found on the aforementioned tomb.}\\*
Dig the grave and let me lie.\\
Glad did I live and gladly die,\\*
\vin And I laid me down with a will.\\!

This be the verse you 'grave for me:\\*
{\it Here he lies where he longed to be;\\
Home is the sailor, home from the sea,\\*
\vin And the hunter home from the hill.}
\end{verse}

\end{document}