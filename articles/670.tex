\documentclass[MAIN]{subfiles}
\begin{document}

\settowidth{\versewidth}{Sans teeth, sans eyes, sans taste, sans everything.}
\begin{verse}[\versewidth]
\poemlines{5}
\vin \vin \vin \vin \vin All the world's a stage,\footnotetext{$\xi$ William Shakespeare (1564 -- 1616). \cite{obev}. These lines are spoken by Jaques in \emph{As You Like It}, II.7.}\\*
And all the men \& women merely players;\\
They have their exits \& their entrances,\\
And one man in his time plays many parts,\\
His acts being seven ages. At first, the infant,\\
Mewling \& puking in the nurse's arms.\\
Then the whining schoolboy, with his satchel\\
And shining morning face, creeping like snail\\
Unwillingly to school. And then the lover,\\
Sighing like furnace, with a woeful ballad\\
Made to his mistress' eyebrow. Then a soldier,\\
Full of strange oaths and bearded like the pard,\\
Jealous in honor, sudden \& quick in quarrel,\\
Seeking the bubble reputation\\
Even in the cannon's mouth. And then the justice,\\
In fair round belly with good capon lined,\\
With eyes severe \& beard of formal cut,\\
Full of wise saws \& modern instances;\\
And so he plays his part. The sixth age shifts\\
Into the lean \& slippered pantaloon,\\
With spectacles on nose \& pouch on side;\\
His youthful hose, well saved, a world too wide\\
For his shrunk shank, and his big manly voice,\\
Turning again toward childish treble, pipes\\
And whistles in his sound. Last scene of all,\\
That ends this strange eventful history,\\
Is second childishness \& mere oblivion,\\*
Sans teeth, sans eyes, sans taste, sans everything.
\end{verse}

\end{document}