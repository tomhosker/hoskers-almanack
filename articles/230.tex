\documentclass[MAIN]{subfiles}
\begin{document}

\settowidth{\versewidth}{Her eyes the glow-worm lend thee;}
\begin{verse}[\versewidth]
\poemlines{5}
Her eyes the glow-worm lend thee;\footnotetext{`The Night Piece, to Julia', Rev Robert Herrick (1591 -- 1674). \cite{norton}. \P A \emph{will-o'-the-wisp} is a phenomenon, which appears as a pale patch of light, sometimes seen by travellers walking through the countryside at night. \P A \emph{slow-worm}, meanwhile, is an archaic name for an adder, i.e. \emph{Vipera berus}.}\\*
The shooting stars attend thee;\\
\vin And the elves also,\\
\vin Whose little eyes glow\\*
Like the sparks of fire, befriend thee.\\!

No will-o'-the-wisp mis-light thee,\\*
Nor snake or slow-worm bite thee;\\
\vin But on, on thy way,\\
\vin Not making a stay,\\*
Since ghost there's none to affright thee.\\!

Let not the dark thee cumber;\\*
What though the moon does slumber?\\
\vin The stars of the night\\
\vin Will lend thee their light,\\*
Like tapers clear without number.\\!

Then \emph{Julia} let me woo thee,\\*
Thus, thus to come unto me;\\
\vin And when I shall meet\\
\vin Thy silv'ry feet,\\*
My soul I'll pour into thee.
\end{verse}

\end{document}