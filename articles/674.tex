\documentclass[MAIN]{subfiles}
\begin{document}

\settowidth{\versewidth}{Knock-kneed, coughing like hags, we cursed through sludge,}
\begin{verse}[\versewidth]
\poemlines{5}
Bent double, like old beggars under sacks,\footnotetext{`Dulce et Decorum Est', Wilfrid Owen (1893 -- 1918). \cite{ptmgmc}. The Latin phrase is from Horace (\emph{Carmina} III.2). The poem is sometimes said to be a response to the poetry of Jessie Pope.}\\*
Knock-kneed, coughing like hags, we cursed through sludge,\\
Till on the haunting flares we turned our backs,\\
And towards our distant rest began to trudge.\\
Men marched asleep. Many had lost their boots,\\
But limped on, blood-shod. All went lame; all blind;\\
Drunk with fatigue; deaf even to the hoots\\*
Of gas-shells dropping softly behind.\\!

`Gas! GAS! Quick, boys!' An ecstasy of fumbling\\*
Fitting the clumsy helmets just in time,\\
But someone still was yelling out and stumbling\\
And flound'ring like a man in fire or lime.\\
Dim through the misty panes \& thick green light,\\*
As under a green sea, I saw him drowning.\\!

In all my dreams before my helpless sight,\\*
He plunges at me, guttering, choking, drowning.\\!

If in some smothering dreams, you too could pace\\*
Behind the wagon that we flung him in,\\
And watch the white eyes writhing in his face,\\
His hanging face, like a devil's sick of sin;\\
If you could hear, at every jolt, the blood\\
Come gargling from the froth-corrupted lungs,\\
Obscene as cancer, bitter as the cud\\
Of vile, incurable sores on innocent tongues --\\
My friend, you would not tell with such high zest\\
To children ardent for some desperate glory\\
The old lie: {\hge Dulce et decorum est\\*
Pro patria mori}.
\end{verse}

\end{document}