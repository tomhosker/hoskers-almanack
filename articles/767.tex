\documentclass[MAIN]{subfiles}

\begin{document}

\begin{center}
\emph{Tune: Who's the Fool Now?}
\end{center}

\bigskip

\settowidth{\versewidth}{`Fill thou the cup and I the can.}
\begin{verse}[\versewidth]
O \emph{Martin} said to his man,\footnotetext{`Who's the Fool Now?', Anon. \cite{old_england}. This song is truly ancient; a version of it apppears in Ravenscroft's \emph{Deuteromelia} (1609). The narrative behind it is a rich man and his servant drinking together. The rich man is drinking considerably larger quantities, and is amusing his servant with improbable tales. \P The `man in the moon' is said to refer here to Henry VIII, and his troubles with Clement VII, although this seems more likely to be just nonsense verse, at least at first glance.}\\*
\vin {\it Fie, man! Fie!}\\
\emph{Martin} said to his man,\\
\vin {\it Who's the fool now?}\\
\emph{Martin} said to his man,\\
`Fill thou the cup and I the can.\\
Thou hast well drunken, man.\\*
Who's the fool now?\\!

I saw the man in the moon\\*
Clouting of St \emph{Peter}'s shoon.\\!

I saw a hare chase a hound\\*
Twenty miles above the ground.\\!

I saw a mouse chase a cat,\\*
Saw a cheese eat a rat.\\!

O \emph{Martin} said to his man,\\*
\vin {\it Fie, man! Fie!}\\
\emph{Martin} said to his man,\\
\vin {\it Who's the fool now?}\\
\emph{Martin} said to his man,\\
`Fill thou the cup and I the can.\\
Thou hast well drunken, man.\\*
Who's the fool now?\\!
\end{verse}
\end{document}
