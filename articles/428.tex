\documentclass[MAIN]{subfiles}
\begin{document}

\settowidth{\versewidth}{In the long, sleepless watches of the night,}
\begin{verse}[\versewidth]
\poemlines{5}
In the long, sleepless watches of the night,\footnotetext{`The Cross of Snow', Prof Henry Longfellow (1807 -- 1882). \cite{norton}. Prof Longfellow survived both of his wives. The first, Elizabeth, died at twenty-two following a miscarriage. The second, Frances, having given him six children, died in an horrific accident; her dress caught fire while she was sealing envelopes with melted wax, and, although Prof Longfellow heroically tried to smother the flames with his own body, she was burned to death. Naturally, the professor was badly burned himself, which perhaps explains the `cross of snow... I wear upon my breast'.}\\*
\vin A gentle face -- the face of one long dead --\\
\vin Looks at me from the wall, where round its head\\
The night-lamp casts a halo of pale light.\\
Here in this room she died; and soul more white\\
\vin Never through martyrdom of fire was led\\
\vin To its repose; nor can in books be read\\
The legend of a life more benedight.\\
There is a mountain in the distant west\\
\vin That, sun-defying, in its deep ravines\\
\vin \vin Displays a cross of snow upon its side.\\
Such is the cross I wear upon my breast\\
\vin These 18 years, through all the changing scenes\\* 
\vin \vin And seasons, changeless since the day she died.
\end{verse}

\end{document}