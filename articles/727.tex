\documentclass[MAIN]{subfiles}
\begin{document}

\settowidth{\versewidth}{In Flanders fields the poppies grow}
\begin{verse}[\versewidth]
\poemlines{5}
In Flanders fields the poppies grow\footnotetext{Lt Col John McCrae (1872 -- 1918). \cite{norton}. The argument of the poem -- that the living should give their lives to avenge the dead -- is clearly stupid. Where would the killing end before the whole world was sacrificed to this quasi-religion of military honour? And indeed the First World War provided a kind of answer to that question. But it remains a fine poem, and was popular with the ordinary soldiers of that most terrible of wars.}\\*
Between the crosses, row on row,\\
\vin That mark our place; and in the sky\\
\vin The larks, still bravely singing, fly\\*
Scarce heard amid the guns below.\\!

We are the dead. Short days ago\\*
We lived, felt dawn, saw sunset glow,\\
\vin Loved \& were loved, and now we lie\\*
\vin \vin \vin In Flanders fields.\\!

Take up our quarrel with the foe:\\*
To you from failing hands we throw\\
\vin The torch; be yours to hold it high.\\
\vin If ye break faith with us who die\\
We shall not sleep, though poppies grow\\*
\vin \vin \vin In Flanders fields.
\end{verse}

\end{document}