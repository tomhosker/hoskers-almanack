\documentclass[MAIN]{subfiles}
\begin{document}

\settowidth{\versewidth}{Five years have passed; five summers, with the length}
\begin{verse}[\versewidth]
\poemlines{5}
Five years have passed; five summers, with the length\footnotetext{`Lines Composed a Few Miles above Tintern Abbey, On Revisiting the Banks of the Wye during a Tour, July 13, 1798', Dr William Wordsworth, Poet Laureate (1770 -- 1850). \cite{norton}.}\\* 
Of five long winters, and again I hear\\
These waters, rolling from their mountain springs\\
With a soft inland murmur. Once again\\
Do I behold these steep \& lofty cliffs,\\
That on a wild secluded scene impress\\
Thoughts of more deep seclusion; and connect\\
The landscape with the quiet of the sky.\\
The day is come when I again repose\\
Here, under this dark sycamore, and view\\
These plots of cottage-ground, these orchard-tufts,\\
Which at this season, with their unripe fruits,\\
Are clad in one green hue, and lose themselves\\
'Mid groves and copses. Once again I see\\
These hedge-rows, hardly hedgerows, little lines\\
Of sportive wood run wild: these pastoral farms,\\
Green to the very door; and wreaths of smoke\\
Sent up, in silence, from among the trees!\\
With some uncertain notice, as might seem\\
Of vagrant dwellers in the houseless woods,\\ 
Or of some hermit's cave, where by his fire\\
The hermit sits alone. These beauteous forms,\\ 
Through a long absence, have not been to me\\
As is a landscape to a blind man's eye:\\
But oft, in lonely rooms, and 'mid the din\\ 
Of towns and cities, I have owed to them,\\
In hours of weariness, sensations sweet,\\
Felt in the blood, and felt along the heart;\\ 
And passing even into my purer mind\\
With tranquil restoration: feelings too\\ 
Of unremembered pleasure: such, perhaps,\\ 
As have no slight or trivial influence\\
On that best portion of a good man's life,\\ 
His little, nameless, unremembered, acts\\
Of kindness and of love. Nor less, I trust,\\ 
To them I may have owed another gift,\\
Of aspect more sublime; that blessed mood,\\ 
In which the burthen of the mystery,\\
In which the heavy and the weary weight\\ 
Of all this unintelligible world,\\
Is lightened: that serene and bless\`ed mood,\\ 
In which the affections gently lead us on,\\
Until, the breath of this corporeal frame\\
And even the motion of our human blood\\
Almost suspended, we are laid asleep\\
In body, and become a living soul:\\
While with an eye made quiet by the power\\ 
Of harmony, and the deep power of joy,\\
We see into the life of things. If this\\ 
Be but a vain belief, yet, O! how oft\\
In darkness and amid the many shapes\\
Of joyless daylight; when the fretful stir\\ 
Unprofitable, and the fever of the world,\\
Have hung upon the beatings of my heart\\
How oft, in spirit, have I turned to thee,\\ 
O sylvan {\sc Wye}! thou wanderer through the woods,\\* 
How often has my spirit turned to thee!\\!

And now, with gleams of half-extinguished thought,\\*
With many recognitions dim and faint,\\
And somewhat of a sad perplexity,\\
The picture of the mind revives again:\\ 
While here I stand, not only with the sense\\
Of present pleasure, but with pleasing thoughts\\ 
That in this moment there is life and food\\
For future years. And so I dare to hope,\\
Though changed, no doubt, from what I was when first\\
I came among these hills; when like a roe\\
I bounded o'er the mountains, by the sides\\
Of the deep rivers, and the lonely streams,\\
Wherever nature led: more like a man\\
Flying from something that he dreads, than one\\ 
Who sought the thing he loved. For nature then\\
(The coarser pleasures of my boyish days\\
And their glad animal movements all gone by)\\ 
To me was all in all. I cannot paint\\
What then I was. The sounding cataract\\ 
Haunted me like a passion: the tall rock,\\ 
The mountain, and the deep and gloomy wood,\\
Their colours and their forms, were then to me\\ 
An appetite; a feeling and a love,\\
That had no need of a remoter charm,\\ 
By thought supplied, not any interest\\
Unborrowed from the eye. That time is past,\\ 
And all its aching joys are now no more,\\
And all its dizzy raptures. Not for this\\
Faint I, nor mourn nor murmur; other gifts\\ 
Have followed; for such loss, I would believe,\\
Abundant recompense. For I have learned\\
To look on nature, not as in the hour\\
Of thoughtless youth; but hearing oftentimes\\ 
The still sad music of humanity,\\
Nor harsh nor grating, though of ample power\\
To chasten and subdue. And I have felt\\
A presence that disturbs me with the joy\\ 
Of elevated thoughts; a sense sublime\\
Of something far more deeply interfused,\\ 
Whose dwelling is the light of setting suns,\\ 
And the round ocean and the living air,\\
And the blue sky, and in the mind of man:\\ 
A motion and a spirit, that impels\\
All thinking things, all objects of all thought,\\ 
And rolls through all things. Therefore am I still\\
A lover of the meadows \& the woods\\
And mountains; and of all that we behold\\ 
From this green earth; of all the mighty world\\ 
Of eye, and ear, both what they half create,\\
And what perceive; well pleased to recognise\\
In nature and the language of the sense\\
The anchor of my purest thoughts, the nurse,\\ 
The guide, the guardian of my heart, and soul\\
Of all my moral being. Nor perchance,\\ 
If I were not thus taught, should I the more\\
Suffer my genial spirits to decay:\\
For thou art with me here upon the banks\\ 
Of this fair river; thou my dearest friend,\\
My dear, dear friend; and in thy voice I catch\\ 
The language of my former heart, and read\\
My former pleasures in the shooting lights\\
Of thy wild eyes. O! yet a little while\\
May I behold in thee what I was once,\\
My dear, dear sister! and this prayer I make,\\ 
Knowing that nature never did betray\\
The heart that loved her; 'tis her privilege,\\ 
Through all the years of this our life, to lead\\ 
From joy to joy: for she can so inform\\
The mind that is within us, so impress\\
With quietness and beauty, and so feed\\
With lofty thoughts, that neither evil tongues,\\
Rash judgments, nor the sneers of selfish men,\\
Nor greetings where no kindness is, nor all\\
The dreary intercourse of daily life,\\
Shall e'er prevail against us, or disturb\\
Our cheerful faith, that all which we behold\\
Is full of blessings. Therefore let the moon\\
Shine on thee in thy solitary walk;\\
And let the misty mountain-winds be free\\
To blow against thee: and, in after years,\\ 
When these wild ecstasies shall be matured\\
Into a sober pleasure; when thy mind\\
Shall be a mansion for all lovely forms,\\
Thy memory be as a dwelling-place\\
For all sweet sounds and harmonies; O! then,\\ 
If solitude, or fear, or pain, or grief,\\
Should be thy portion, with what healing thoughts\\
Of tender joy wilt thou remember me,\\
And these my exhortations! Nor, perchance\\ 
If I should be where I no more can hear\\
Thy voice, nor catch from thy wild eyes these gleams\\ 
Of past existence -- wilt thou then forget\\
That on the banks of this delightful stream\\
We stood together; and that I, so long\\
A worshipper of nature, hither came\\
Unwearied in that service: rather say\\ 
With warmer love -- O! with far deeper zeal\\
Of holier love. Nor wilt thou then forget,\\
That after many wanderings, many years\\
Of absence, these steep woods and lofty cliffs,\\ 
And this green pastoral landscape, were to me\\
More dear, both for themselves and for thy sake!\\
\end{verse}

\end{document}