\documentclass[MAIN]{subfiles}

\begin{document}

\begin{center}
\emph{Tune: The Postman's Knock}
\end{center}

\bigskip

\settowidth{\versewidth}{When you open the door to his loud rat-tat}
\begin{verse}[\versewidth]
What a wonderful man the postman is\footnotetext{\cite{albion_band}. The words to this sonh are said to have been written by one L M Thornton, flourishing in 1860.}\\*
As he hastens from door to door!\\
What medley of news his hands contain\\
For high, low, rich, \& poor!\\
In many's the face the joy he can trace,\\
In many's the grief he can see,\\
When you open the door to his loud rat-tat\\*
And his quick delivery.\\!

{\it Every morning as true as the clock\\*
Somebody hears the postman's knock!}\\!

Number 1 he presents with news of a birth;\\*
With tidings of death, \textnumero 4.\\
At 13 a bill of terrible length\\
He drops through the hole in the door.\\
Now a cheque or an order for 15 he leaves\\
In 16 his presence to prove,\\
While 17 doth an acknowledgement get,\\*
And 18 a letter of love.\\!

And the mail must get through\\*
Whatever the hazards or odds.\\
This low man of letters just peddles on through\\
Pursued by a pack of wild dogs.\\
But ease \& complaining whatever the trial\\
Or beating he never retreats,\\
For you get a free bag \& a hat with a badge\\*
And it's better than walking the streets.\\!

{\it Every morning as true as the clock\\*
Somebody hears the postman's knock!}
\end{verse}
\end{document}
