\documentclass[MAIN]{subfiles}
\begin{document}

\settowidth{\versewidth}{\vin Tell her, that her sweet tongue was wont to make me mirth.}
\begin{verse}[\versewidth]
\poemlines{5}
Unhappy verse, the witness of my unhappy state,\footnotetext{`Iambicum Trimetrum', Edmund Spenser (1552 -- 1559). \cite{newlove}. The title of this poem is a mystery to the Almanackist, who would be grateful if anyone could explain it to him. The word \emph{immerito} is Italian for \emph{undeserved}, although in this poem it seems to be used primarily as a man's name.}\\*
\vin Make thy self flutt'ring wings of thy fast flying\\
\vin Thought, and fly forth unto my love, wheresoever she be:\\
Whether lying restless in heavy bed, or else\\
\vin Sitting so cheerless at the cheerful board, or else\\
\vin Playing alone careless on her heavenly virginals.\\
If in bed, tell her, that my eyes can take no rest:\\
\vin If at board, tell her, that my mouth can eat no meat:\\ 
\vin If at her virginals, tell her, I can hear no mirth.\\
Asked why say: waking love suffereth no sleep:\\
\vin Say that raging love doth appal the weak stomach:\\
\vin Say that lamenting love marreth the musical.\\
Tell her, that her pleasures were wont to lull me asleep:\\
\vin Tell her, that her beauty was wont to feed mine eyes:\\
\vin Tell her, that her sweet tongue was wont to make me mirth.\\
Now do I nightly waste, wanting my kindly rest:\\
\vin Now do I daily starve, wanting my lively food:\\
\vin Now do I always die, wanting thy timely mirth.\\
And if I waste, who will bewail my heavy chance?\\
\vin And if I starve, who will record my curs\`ed end?\\*
\vin And if I die, who will say: `This was \emph{Immerito}'?
\end{verse}

\end{document}