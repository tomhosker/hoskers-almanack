\documentclass[MAIN]{subfiles}
\begin{document}

\settowidth{\versewidth}{\vin Fed with cold and usurous hand?}
\begin{verse}[\versewidth]
\poemlines{5}
Is this a holy thing to see,\footnotetext{$\xi$ `Holy Thursday', William Blake (1757 -- 1827). \cite{blakea}. The Almanackist has excised the second verse. This is from \emph{Songs of Experience}; Blake wrote another poem called `Holy Thursday' which he published in \emph{Songs of Innocence}.}\\*
\vin In a rich and fruitful land,\\
Babes reduced to misery,\\*
\vin Fed with cold \& usurous hand?\\!

And their sun does never shine.\\*
\vin And their fields are bleak \& bare.\\
And their ways are filled with thorns.\\*
\vin It is eternal winter there.\\!

For where'er the sun does shine,\\*
\vin And where'er the rain does fall:\\
Babe can never hunger there,\\*
\vin Nor poverty the mind appal.
\end{verse}

\end{document}