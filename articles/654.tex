\documentclass[MAIN]{subfiles}
\begin{document}

\settowidth{\versewidth}{I met a traveller from an antique land}
\begin{verse}[\versewidth]
I met a traveller from an antique land\footnotetext{`Ozymandias of Egypt', Percy Shelley (1792 -- 1822). \cite{treasury}. Shelley seems to have written this sonnet as one half of a sonnet-writing competition with his friend Horace Smith, who published a very similar, if clearly inferior, poem in the same journal a month later. \P Professor Holloway, in his introduction to the \emph{Oxford Book of Local Verses}, highlights the double meaning in the eleventh line: `Time, the poet intimates, invites the proud of a later age, as they gaze upon the forgotten ruins of that spurious grandeur, to despair in a deeper sense.'}\\
Who said: `Two vast \& trunkless legs of stone\\
Stand in the desert. Near them on the sand,\\
Half sunk, a shattered visage lies, whose frown\\
And wrinkled lip \& sneer of cold command\\
Tell that its sculptor well those passions read\\
Which yet survive, stamped on these lifeless things,\\
The hand that mocked them and the heart that fed.\\
And on the pedestal these words appear:\\
{\it My name is \emph{Ozymandias}, king of kings:\\
Look on my works, ye mighty, and despair!}\\
Nothing beside remains: round the decay\\
Of that colossal wreck, boundless \& bare,\\*
The lone \& level sands stretch far away.'
\end{verse}

\end{document}