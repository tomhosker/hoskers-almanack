\documentclass[MAIN]{subfiles}
\begin{document}

\settowidth{\versewidth}{You meaner beauties of the night,}
\begin{verse}[\versewidth]
\poemlines{5}
You meaner beauties of the night,\footnotetext{`Elizabeth of Bohemia', Sir Henry Wotton (1568 -- 1639). \cite{treasury}. The poem would seem to be dedicated to Queen Elizabeth, wife of Frederick, King of Bohemia; alas, any further detail is lacking.}\\*
\vin Which poorly satisfy our eyes\\
More by your number than your light,\\
\vin You common people of the skies --\\*
What are you, when the moon shall rise?\footnotetext{5. Where Palgrave reads `Moon', the best texts consulted read `Sun'; but the Almanackist finds `Moon' more pleasing.}\\!

Ye violets that first appear,\\*
\vin By your pure purple mantles known\\
Like the proud virgins of the year,\\
\vin As if the spring were all your own --\\*
What are you, when the rose is blown?\\!

Ye curious chanters of the wood\\*
\vin That warble forth dame nature's lays,\\
Thinking your passions understood\\
\vin By your weak accents -- what's your praise\\*
When \emph{Philomel} her voice doth raise?\\!

So when my mistress shall be seen\\*
\vin In sweetness of her looks \& mind,\\
By virtue first, then choice, a queen,\\
\vin Tell me, if she were not designed\\*
Th' eclipse \& glory of her kind?
\end{verse}

\end{document}