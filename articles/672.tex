\documentclass[MAIN]{subfiles}
\begin{document}

\settowidth{\versewidth}{Curled in her court false locks of living hair,}
\begin{verse}[\versewidth]
\poemlines{5}
As I beheld a winter's evening air,\footnotetext{`Another: A Black patch on Lucasta's Face', Colonel Richard Lovelace (1617 -- 1657). \cite{obev}.}\\*
Curled in her court false locks of living hair,\\*
Buttered with jessamine the sun left there.\\!

Galliard \& clinquant she appeared to give,\\*
A serenade or ball to us that grieve,\\*
And teach us {\hge a la mode} more gently live.\\!

But as a moor, who to her cheeks prefers\\*
White spots, t'allure her black idolaters,\\ *
Me thought she looked all o'er bepatched with stars.\\!

Like the dark front of some ethiopian queen,\\*
Veiled all o'er with gems of red, blue, green,\\*
Whose ugly night seemed masked with days' skreen.\\!

Whilst the fond people offered sacrifice\\* 
To sapphires, 'stead of veins \& arteries,\\*
And bowed unto the diamonds, not her eyes.\\!

Behold \emph{Lucasta}'s face, how't glows like noon!\\*
A sun entire is her complexion,\\*
And formed of one whole constellation.\\!

So gently shining, so serene, so clear,\\*
Her look doth universal nature cheer;\\*
Only a cloud or two hangs here \& there.
\end{verse}

\end{document}