\documentclass[MAIN]{subfiles}
\begin{document}

\settowidth{\versewidth}{But since thy finished labour hath possessed}
\begin{verse}[\versewidth]
\poemlines{5}
My dearest dust, could not thy hasty day\footnotetext{Catherine Dyer, Lady Dyer (d. 1654). \cite{newlove}. Lady Dyer had this remarkable epitaph inscribed on the monument of her late husband, Sir William Dyer (1583 -- 1621), which can be found in St Denys' Church in the village of Colmworth, Bedfordshire. This sonnet is actually just the second half of the complete epitaph. \P There is some ambiguity in the twelfth line: some books gives `my blood grows cold', while others give `my beloved grows' cold. The Almanackist is no scholar of seventeenth century English orthography, but he has seen the original monument himself, and can report that it reads `MY BLOVD GROWES COLD', and thus he concludes that either interpretation may be correct.}\\* 
Afford thy drowsy patience leave to stay\\
One hour longer: so that we might either\\
Sit up, or gone to bed together?\\
But since thy finished labour hath possessed\\
Thy weary limbs with early rest,\\
Enjoy it sweetly: and thy widow bride\\
Shall soon repose her by thy slumb'ring side.\\ 
Whose business, now, is only to prepare\\
My nightly dress, and call to prayer:\\
Mine eyes wax heavy and the day grows old.\\
The dew falls thick; my blood grows cold.\\
Draw, draw the clos\`ed curtains: and make room:\\*
My dear, my dearest dust; I come, I come.
\end{verse}

\end{document}