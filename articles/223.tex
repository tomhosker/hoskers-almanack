\documentclass[MAIN]{subfiles}
\begin{document}

\settowidth{\versewidth}{Two roads diverged in a yellow wood,}
\begin{verse}[\versewidth]
\poemlines{5}
Two roads diverged in a yellow wood,\footnotetext{`The Road Not Taken', Robert Frost, Poet Laureate of the United States, Poet Laureate of Vermont (1874 -- 1963). \cite{norton}. Frost wrote this poem as a kind of parody, based on an in-joke between himself and his friend Edward Thomas (or, at least, Frost used to claim as much, but poets often have mixed feelings towards their most famous works).}\\*
\vin And sorry I could not travel both\\
And be one traveler, long I stood\\
And looked down one as far as I could\\*
\vin To where it bent in the undergrowth;\\!

Then took the other, as just as fair,\\*
\vin And having perhaps the better claim,\\
Because it was grassy \& wanted wear;\\
Though as for that the passing there\\*
\vin Had worn them really about the same,\\!

And both that morning equally lay\\*
\vin In leaves no step had trodden black.\\
O I kept the first for another day!\\
Yet knowing how way leads on to way,\\*
\vin I doubted if I should ever come back.\\!

I shall be telling this with a sigh\\*
\vin Somewhere ages \& ages hence:\\
Two roads diverged in a wood, and I --\\
I took the one less traveled by,\\*
\vin And that has made all the difference.
\end{verse}

\end{document}