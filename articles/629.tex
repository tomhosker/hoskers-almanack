\documentclass[MAIN]{subfiles}
\begin{document}

\settowidth{\versewidth}{A reeling road, a rolling road, that rambles round the shire,}
\begin{verse}[\versewidth]
\poemlines{5}
Before the roman came to {\sc Rye} or out to {\sc Severn} strode,\footnotetext{`The Rolling English Road', Gilbert Chesterton, Knight (1874 -- 1936). \cite{oxfordlarkin}. Anyone with a rudimentary grasp of British geography could tell you that the routes suggested in the last line of each verse are unlikely to be the most direct.}\\*
The rolling english drunkard made the rolling english road.\\
A reeling road, a rolling road, that rambles round the shire,\\
And after him the parson ran, the sexton \& the squire;\\
A merry road, a mazy road, and such as we did tread\\*
The night we went to {\sc Birmingham} by way of {\sc Beachy Head}.\\!

I knew no harm of \emph{Bonaparte} and plenty of the squire,\\*
And for to fight the frenchman I did not much desire;\\
But I did bash their bayonets because they came arrayed\\
To straighten out the crooked road an english drunkard made,\\
Where you \& I went down the lane with ale-mugs in our hands,\\*
The night we went to {\sc Glastonbury} by way of {\sc Goodwin Sands}.\\! 

His sins they were forgiven him; or why do flowers run\\*
Behind him; and the hedges all strengthening in the sun?\\
The wild thing went from left to right and knew not which was which,\\
But the wild rose was above him when they found him in the ditch.\\
God pardon us, nor harden us; we did not see so clear\\*
The night we went to {\sc Bannockburn} by way of {\sc Brighton Pier}.\\! 

My friends, we will not go again or ape an ancient rage,\\*
Or stretch the folly of our youth to be the shame of age,\\
But walk with clearer eyes and ears this path that wandereth,\\
And see undrugged in evening light the decent inn of death;\\
For there is good news yet to hear and fine things to be seen,\\*
Before we go to paradise by way of {\sc Kensal Green}.
\end{verse}

\end{document}