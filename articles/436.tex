\documentclass[MAIN]{subfiles}
\begin{document}

\settowidth{\versewidth}{My prime of youth is but a frost of cares;}
\begin{verse}[\versewidth]
\poemlines{5}
My prime of youth is but a frost of cares;\footnotetext{`Tichborne's Lament', Chidiock Tichborne (1562 -- 1586). \cite{treasury}. The ultimate written source for this poem is a letter which Tichborne wrote to his wife on the night before he was hanged, drawn and quartered for his part in a conspiracy against Elizabeth I. Tichborne was part of the same family which provided the fourteen Tichborne baronets (of Tichborne in the County of Hampshire) who held the title from its creation in 1621 until its extinction in 1968; he was also a distant cousin of Henry Tichborne, 1st Baron Ferrard and 1st Baronet (of Beaulieu in the County of Meath), who sadly left no heirs to his titles. \P The `glass' in the penultimate line refers to an hourglass, rather than a drinking vessel.}\\*
\vin My feast of joy is but a dish of pain;\\
My crop of corn is but a field of tares;\\
\vin And all my good is but vain hope of gain;\\
The day is past, and yet I saw no sun,\\*
And now I live, and now my life is done.\\!

My tale was heard and yet it was not told;\\*
\vin My fruit is fallen, and yet my leaves are green;\\
My youth is spent and yet I am not old;\\
\vin I saw the world and yet I was not seen;\\
My thread is cut and yet it is not spun,\\*
And now I live, and now my life is done.\\!

I sought my death and found it in my womb;\\*
\vin I looked for life and saw it was a shade;\\
I trod the earth and knew it was my tomb,\\
\vin And now I die, and now I was but made;\\
My glass is full, and now my glass is run,\\*
And now I live, and now my life is done.
\end{verse}

\end{document}