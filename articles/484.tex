\documentclass[MAIN]{subfiles}
\begin{document}

\settowidth{\versewidth}{The piers are pummelled by the waves;}
\begin{verse}[\versewidth]
\poemlines{5}
The piers are pummelled by the waves;\footnotetext{`The Fall of Rome', Prof. Wystan Auden (1907 -- 1973). \cite{audena}. Prof. Auden dedicated the poem to Cyril Connolly. Each verse seems to consider a reason frequently given by historians for the collapse of the Roman Empire. In particular, the last verse concerns the following theory: changes in the climate forced the migration of certain species (amongst them, reindeer) on the steppes of Eastern Europe, which in turn forced the migration of those tribes which depended on said species. This led to a chain of tribal migrations, culminating in the barbarian invasions of Late Antiquity which brought about the Empire's demise.}\\*
In a lonely field the rain\\
Lashes an abandoned train;\\*
Outlaws fill the mountain caves.\\!

Fantastic grow the evening gowns;\\*
Agents of the Fisc pursue\\
Absconding tax-defaulters through\\*
The sewers of provincial towns.\\!

Private rites of magic send\\*
The temple prostitutes to sleep;\\
All the literati keep\\*
An imaginary friend.\\!

Cerebrotonic \emph{Cato} may\\*
Extol the ancient disciplines,\\
But the muscle-bound marines\\*
Mutiny for food \& pay.\\!

Caesar's double bed is warm\\*
While an unimportant clerk\\
Writes, `I do not like my work,'\\*
On a pink official form.\\!

Unendowed with wealth or pity,\\*
Little birds with scarlet legs,\\
Sitting on their speckled eggs,\\*
Eye each flu-infected city.\\!

Altogether elsewhere, vast\\*
Herds of reindeer move across\\
Miles \& miles of golden moss,\\*
Silently and very fast.
\end{verse}

\end{document}