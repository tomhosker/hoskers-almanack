\documentclass[MAIN]{subfiles}

\begin{document}

\begin{center}
\emph{Tune: The Goodman}
\end{center}

\bigskip

\settowidth{\versewidth}{`It's a cow! It's a cow!' cried the goodman's wife,}
\begin{verse}[\versewidth]
The goodman he came home one night.\footnotetext{Anon. \cite{rusby_underneath}. This song seems to have its roots in an old Scottish song, Child Ballad \# 274, but from there it has grown into thousands of forms, in a number of European languages.}\\*
The goodman home came he.\\
There he spied an old saddle horse\\
Where no horse should there be.\\
`It's a cow! It's a cow!' cried the goodman's wife,\\
`A cow, just a cow, can't you see?'\\
Far have I ridden, and much I've seen,\\*
But a saddle on a cow has never been.\\!

The goodman he came home one night.\\*
The goodman home came he.\\
There he spied a powdered wig\\
Where no wig should there be.\\
`It's a hen! It's a hen!' cried the goodman's wife,\\
`A hen, just a hen, can't you see?'\\
Far have I ridden, and much I've seen,\\*
But powder on a hen has never been.\\!

The goodman he came home one night.\\*
The goodman home came he.\\
There he spied a riding coat\\
Where no coat should there be.\\
`It's sheets, just sheets!' cried the goodman's wife,\\
`Sheets, just sheets, can't you see?'\\
Far have I ridden, and much I've seen,\\*
But buttons on a sheet has never been.\\!

The goodman he came home one night.\\*
The good man home came he.\\
There he spied a handsome man\\
Where no man should there be.\\
`It's the maid! It's the maid!' cried the goodman's wife,\\
`The milking maid, can't you see?'\\
Far have I ridden, and much I've seen,\\*
But a beard on a maid has never been.
\end{verse}
\end{document}
