\documentclass[MAIN]{subfiles}
\begin{document}

\settowidth{\versewidth}{Black lamps from the halls of \emph{Dis}, burning dark-blue,}
\begin{verse}[\versewidth]
\poemlines{5}
Not every man has gentians in his house\footnotetext{`Bavarian Gentians', David Lawrence (1885 -- 1930). \cite{norton}. It comes as no surprise that Lawrence wrote this poem only a few months before his own painfully-anticipated death from tuberculosis.}\\*
In soft september, at slow, sad michaelmas.\\!

Bavarian gentians, big \& dark, only dark\\*
Darkening the daytime torch-like with the smoking blueness of \emph{Pluto}'s gloom,\\
Ribbed \& torch-like, with their blaze of darkness spread blue\\
Down flattening into points, flattened under the sweep of white day,\\
Torch-flower of the blue-smoking darkness, \emph{Pluto}'s dark-blue daze,\\
Black lamps from the halls of \emph{Dis}, burning dark-blue,\\
Giving off darkness, blue darkness, as \emph{Demeter}'s pale lamps\\
Give off light,\\*
Lead me then; lead me the way.\\!

Reach me a gentian; give me a torch!\\*
Let me guide myself with the blue, forked torch of a flower\\
Down the darker \& darker stairs, where blue is darkened on blueness,\\
Even where \emph{Persephone} goes, just now, from the frosted september\\
To the sightless realm where darkness is awake upon the dark\\
And \emph{Persephone} herself is but a voice\\
Or a darkness invisible enfolded in the deeper dark\\
Of the arms plutonic, and pierced with the passion of dense gloom,\\*
Among the splendour of torches of darkness, shedding darkness on the lost bride \& her groom.
\end{verse}

\end{document}